\chapter{Anhang}
\section{Dokumentierter Programmcode zu Aufgabe 1}
\lstset{language=MATLAB}
\begin{lstlisting}
clear all
close all
%% m<=n
n=9; %Grad 
m=0; %Ordnung

delta = pi/60; %Berechnung des Breiten- L�ngengrads/Umrechnung
theta = 0 : delta : pi;
phi = 0 : 2*delta : 2*pi;
[phi, theta] = meshgrid(phi,theta);

Pmn = legendre(n, cos(theta(:,1)), 'sch');
Pmn = Pmn(m+1,:)'; %Berechnung der Legendre Polynome mittels 
			   %Funktion 'sch' (Schmidtnormalisierung)

for i = 1: size(phi,1) %Schleife um Legendre Polynome �ber Phi
    Pmnphi(:,i) = Pmn(:);
end;

F = Pmnphi.*(cos(m*phi)+sin(m*phi)); %Berechnung nach                   							%gegebener Formel
%Graphische Darstellung
figure
[x,y,z]=sphere(length(F));
surf(x,y,z,F)
axis square
xlabel('x')
ylabel('y')
zlabel('z')
title('Zonales Legendre Polynom mit n=9 und m=0')
colorbar;
%% m<=n
n=6; %Grad 
m=6; %Ordnung

delta = pi/60;
theta = 0 : delta : pi;
phi = 0 : 2*delta : 2*pi;
[phi, theta] = meshgrid(phi,theta);

Pmn = legendre(n, cos(theta(:,1)), 'sch');
Pmn = Pmn(m+1,:)';

for i = 1: size(phi,1)
    Pmnphi(:,i) = Pmn(:);
end;

F = Pmnphi.*(cos(m*phi)+sin(m*phi));

figure
[x,y,z]=sphere(length(F));
surf(x,y,z,F)
axis square
xlabel('x')
ylabel('y')
zlabel('z')
title('Sektorales Legendre Polynom mit n=m=6')
colorbar;

%% m<=n
n=9; %Grad 
m=3; %Ordnung

delta = pi/60;
theta = 0 : delta : pi;
phi = 0 : 2*delta : 2*pi;
[phi, theta] = meshgrid(phi,theta);

Pmn = legendre(n, cos(theta(:,1)), 'sch');
Pmn = Pmn(m+1,:)';

for i = 1: size(phi,1)
    Pmnphi(:,i) = Pmn(:);
end;

F = Pmnphi.*(cos(m*phi)+sin(m*phi));

figure
[x,y,z]=sphere(length(F));
surf(x,y,z,F)
axis square
xlabel('x')
ylabel('y')
zlabel('z')
title('Tesserales Legendre Polynom mit n=9 und m=3')
colorbar;

\end{lstlisting}
\newpage
\section{Dokumentierter Programmcode zu Aufgabe 2}
\lstset{language=MATLAB}
\begin{lstlisting}
clear all
close all

%% Einlesen des Datenfiles 
filename = 'osu89a-mod.gfc';
[A,delimiterOut]=importdata(filename)

gamma = 6.67384E-11;% Gravitationskonstante
M = 5.974E24;       % Erdmasse
r = 6371000.;       % mittlerer Erdradius
%a = 149.6E6;        % grosse Halbachse
V = gamma*M/r;      % Vorfaktor
a = 0.6378136460E+07;

delta = pi/100;
theta = 0 : delta : pi;     % Breite
phi = 0 : 2*delta : 2*pi;   % L�nge
%Anlegen der Koeffizienenmatrizen i  denen jedem Grad7Ordnung Koeffizienten
%zuegeordnet werden
for line_i=1:length(A.data)
    Cnm(A.data(line_i,1)+1,A.data(line_i,2)+1)=A.data(line_i,3);
    Snm(A.data(line_i,1)+1,A.data(line_i,2)+1)=A.data(line_i,4);
end

%Korrekturen
Cnm(3,1) = Cnm(3,1) + 0.108262982131 * 10^(-2)/sqrt(5);
Cnm(5,1) = Cnm(5,1) - 0.237091120053 * 10^(-5)/sqrt(9);
Cnm(7,1) = Cnm(7,1) + 0.608346498882 * 10^(-8)/sqrt(13);
Cnm(9,1) = Cnm(9,1) - 0.142681087920 * 10^(-10)/sqrt(17);
Cnm(11,1) = Cnm(11,1) + 0.121439275882 * 10^(-13)/sqrt(21);

n=200; %maximale Anzahl der Koeffizienten
yy=zeros(length(theta));
xy=zeros(length(theta));
asdf(1:length(phi),1:length(phi))=0;
for n_i=0:n %Schleife �ber n
    %Berechnung der Legendre Polynome (siehe Aufg.1)
    test=n_i
    Pnm = legendre(n_i,cos(theta),'sch'); 
    for phi_i=1:length(phi)
       for m_i=0:n_i
          %Berechnung des Potenzials nach Formel 2.1
            tmp(m_i+1,:) = Pnm(m_i+1,:).*(Cnm(n_i+1,m_i+1)*...
                cos(m_i*phi(phi_i))+Snm(n_i+1,m_i+1)*sin(m_i*phi(phi_i))); 
       end
       %Anlegen der Summen
       asdf(phi_i,:)=sum(tmp);
       xy(phi_i,:)=xy(phi_i,:)+asdf(phi_i,:);
    end
    qwer(:,:) = xy(:,:)*((a/r)^n_i);
    yy(:,:)=yy(:,:)+qwer(:,:);
end
yy(:,:)=yy(:,:)*V;
yy=yy';

%Normierung der Colourbar
delta_value=min(min(yy));
 %U=U./delta_value-1; Gr��enordnung der Abweichung
 yy=yy-delta_value;
 delta_value
 
 
%Graphische Darstellungen
[phiplot,thetaplot]=meshgrid(phi,theta);
figure(1)
%contourf(theta,phi, yy, 100, 'linestyle','None')
contourf(phiplot*180/pi,thetaplot*180/pi, yy, 100,'linestyle','None')
set(gca,'YDir','reverse')
colorbar
%axissquare
title('Schwerepotential der Erde mit nmax=200')
xlabel('Phi');
ylabel('Theta');


\end{lstlisting}

