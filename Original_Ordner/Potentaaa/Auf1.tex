\chapter{Aufgabe 1}
Im ersten Aufgabenteil des �bungsblattes sollen drei S�tze Legendre Polynome als Funktion der geographischen L�nge und Breite berechnet und programmiert werden. Die jeweiligen Polynome unterscheiden sich durch verschiedene Ordnungen n und Grade m, wobei f�r diese folgende Bedingung gilt: $m \le n$.\\
Die in den anschlie�enden Darstellungen berechneten Polynome definieren sich aus folgender Beziehung:

\begin{equation}
(A_{n,m}cosm\Phi + B_{n,m}sinm\Phi)*P^{m}_{n}(\theta)
\end{equation

Laut Aufgabenstellung gilt f�r die Koeffizienten der Zusammenhang: $A_{n,m}=B_{n,m}=1$
Da die Legendre-Polynome auf der Verwendung beliebig orthogonaler Funktionen basieren haben wir f�r die Normalisierung der zonalen Kugelfunktionen eine Schmidt-Normalisierung mittels der Matlabfunktion \textit{'sch'} durchgef�hrt, sodass aus der Basis des aufgespannten Vektorraums eine Orthnormalbasis konstruiert wird. Durch die Normalisierung wird des Weiteren eine Wichtung der Koeffizienten in einem Intervall von $[-1,1]$vorgenommen.\\
Die nachfolgenden Darstellungen zeigen die Ergebnisse der programmierten Legendre-Polynome. Der entsprechend dokumentierte Programmcode ist dem Anhang A zu entnehmen.

Die erste Abbildung zeigt eine zonale Darstellung der Legendre-Polynome. In diesem Fall ist die Ordnung des Polynoms stets durch m=0 definiert und der Grad des Polynoms variierbar (hier n=9). Da Polynom $P^{0}_{9}$ definiert sich �ber 9 Nullstellen und ist unabh�ngig von den L�ngengraden ($\theta$) (da m=0). Verschiedene Beispiele zeigen, dass durch Erh�hung der Koeffizienten m und n die beschriebenen Gebiete der Funktion kleiner 'gef�chert' sind. 
Die zweite Abbildung f�r $P^{6}_{6}$ ist sektoriell, das hei�t der Grad des Polynoms entspricht seiner Ordnung sodass gilt: m=n=6. In diesem Fall ist das Polynom unabh�ngig von dem Breitengrad und durch 12 Nullstellen definiert. Die dritte Graphik veranschaulicht die allgemeine Kugelfl�chenfunktion $P^{3}_{9}$; eine tesserale Darstellung der unterschiedlichen Koeffizienten. 
 
