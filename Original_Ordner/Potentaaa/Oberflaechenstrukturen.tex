\chapter{Oberfl�chenstrukturen}
Die Oberfl�che des Mondes umfasst 38$km^{2}$ und zeichnet sich durch seine hemisph�rische Gegens�tzlichkeit zwischen der erabgewandten (far side) und erdzugewandten Seite (near side) aus.
So wird die erdzugewandte Seite des Mondes durch die Maria \textit{(Abb.1a)} charakterisiert, welche 31,2$\%$ dieser Fl�che einnehmen. Auf der erdabgewandten Seite sind hingegen die highlands \textit{(Abb.1b)} vorherrschend, die dort 45$\%$ der Oberfl�che bedecken.
Marias sind topographisch niedrige Gebiete, die auch als Tiefebenen bezeichnet werden. Dabei handelt es sich um fast ebene Fl�chen, die entstandene Impaktbecken aus der Fr�hzeit mit basaltischen Lavadecken verf�llt haben.
Die Krustenm�chtigkeit der near side mit maximal 60 km ist deutlich geringer als die der far side und beg�nstigt demensprechende Magmaaustritte an der Oberfl�che, wodurch sich die gr��ere Fl�chenbedeckung der Maria auf der erdzugewandten Seite begr�ndet.\\
Die durch Druckabgabe der Partialschmelzen gebildeten Mare-Basalte weisen anhand ihrer chemischen Zusammensetzung gro�e �hnlichkeiten mit den terrestrischen Basalten der ozeanischen Kruste auf. Das gr��te mit Mare-Basalten verf�llte Becken ist das Imbrium-Becken mit einer Fl�chenausdehung von 830.000 $km^{2}$ und einem Durchmesser von 1123 km im zentral-n�rdlichen Teil auf der near side.\\ 
Die Gegens�tzlichkeit der Maria auf der near side umfasst die auf der far side dominierenden highlands. In diesem Fall handelt es sich um topographisch erh�hte Gebiete die auf der einen Seite von mehreren hunderte Kilometer langen T�lern durchzogen sind und andererseits aus bis zu 10 km hohen Gebirgen bestehen. �ber die Entstehung der highlands existieren mehrere Theorien. Demnach handelt es sich um Reste von Kraterw�nden oder die Formation der highlands erfolgte durch die Komprimierung des Mondes w�hrend der Abk�hlungsphase, wodurch sich die Kruste an der Oberfl�che zu Faltengebirgen aufw�lbte. Die drei Hauptgesteinstypen der Highlands �hneln in ihrem Gef�ge den Plutoniten der Erde und umfassen Anorthosite, KREEP-Basalte \textit{(kristalline rare earth-elements)} und magnesiumreiche Gesteine die sich aus den Anorthositen bilden und �berwiegend mit Olivin und Pyroxen angereichert sind. Aus Isotopensystematiken von Mondanorthositen wurde ein Alter von etwa 4,4 Mrd. Jahren datiert. Dementsprechend stimmt die Altersdatierung der Highland-Gesteine mit dem Bildungsalter der ersten Kruste und dem Kristallisationszeitpunkt des urspr�nglichen Magmaozeans �berein.
