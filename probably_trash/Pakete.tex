%%%%%%%%%%%%%%%%%%%%%%%%%%%%%%%%%%%%%%%%%%%%%%%%%%%%%%%%%%%%%%%%%%%%%%%%%%%%%%%%
%Einbindung von Paketen

%% Deutsche Anpassungen %%%%%%%%%%%%%%%%%%%%%%%%%%%%%%%%%%%%%
\usepackage[ngerman]{babel}
\usepackage[T1]{fontenc}
\usepackage[ansinew]{inputenc}
\usepackage[babel,german=quotes]{csquotes} 
\usepackage{pdfpages}					 % Deutsche Anf�hrungszeichen
\usepackage{mathptmx} 
\usepackage{listings} 
\usepackage{graphicx}                                     % Times New Roman Standard-Schrift mit Mathematik-Paket
%\usepackage[latin1]{inputenc}                               % Eingabekodierung & Unterst�tzung von Umlauten (�,�,�)

\usepackage{lmodern} 										%Type1-Schriftart f�r nicht-englische Texte

%% Packages f�r Formeln %%%%%%%%%%%%%%%%%%%%%%%%%%%%%%%%%%%%%
\usepackage{amsmath}
\usepackage{amsthm}
\usepackage{amsfonts}
\usepackage{amssymb}
\usepackage{mathtools}

%Symbole
\usepackage{marvosym}                                       % Zus�tzliche Symbole (u.A. Euro)
\usepackage{latexsym}                                       % Zus�tzliche mathematische Symbole (11)
\usepackage{dsfont}

%% Zeilenabstand %%%%%%%%%%%%%%%%%%%%%%%%%%%%%%%%%%%%%%%%%%%%
\usepackage{setspace}
\singlespacing        %% 1-zeilig (Standard)
%\onehalfspacing       %% 1,5-zeilig
%\doublespacing        %% 2-zeilig

%Grafische Umgebung
\usepackage{color}                                          % Erm�glicht farbige Texte
\usepackage{epic}                                           % Picture-Umgebung: Einbinden von .pic-Grafiken
\usepackage{eepic}                                          % Erweiterung f�r Picture-Umgebung
%\usepackage{epsfig}
%\usepackage[pdftex]{graphicx}
\usepackage{epstopdf}                                      % Einbinden von Grafiken
\usepackage{subfigure}                                      % Unterabbildungen mit eigenen Unterschriften
\usepackage{PicIns}											% Einbinden von grafiken direkt in den Text bzw. neben Abs�tze

%Tabellen
\usepackage{longtable}                                      % Paket f�r Tabellen, die �ber mehrere Seiten gehen
\usepackage{multicol}                                       % Paket f�r Text in mehreren Spalten
\usepackage{rccol}                                          % Spaltenausrichtung am Komma
\usepackage{booktabs}                                       % Paket f�r toprule/midrule/bottomrule
\usepackage{tabularx}

%Sonstige Pakete
\usepackage{fancyhdr}                                       % Paket zur Gestaltung von Kopf- und Fu�zeile
%\usepackage[breaklinks=true,  colorlinks]{hyperref}         % Links in PDf Dokumenten erzeugen
\usepackage[intoc]{nomencl}                                 % Erstellung eines Formelverzeichnisses
\usepackage{array}                                          % Erstellung von Arrays
\usepackage{setspace}                                       % Paket um Zeilenabstand zu �ndern
\usepackage{caption}                                        % Paket f�r Captions in Tabellen und Bildern
\usepackage[figuresright]{rotating}                         % Paket um Tabellen, Bilder zu drehen (zum rechten Rand gedreht)
%\usepackage[mediumqspace,thickqspace]{SIunits}			% Paket um SI Units zu verwenden
\usepackage{makeidx}								% Paket zum Einbinden eines Stichwortverzeichnisses
\usepackage{pdfpages}